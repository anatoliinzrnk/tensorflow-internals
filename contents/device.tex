\begin{savequote}[45mm]
\ascii{Any fool can write code that a computer can understand. Good programmers write code that humans can understand.}
\qauthor{\ascii{- Martin Flower}}
\end{savequote}

\chapter{设备} 
\label{ch:device}

\begin{content}

\end{content}

\section{设备规范}

\begin{content}

设备规范\ascii{(Device Specification)}用于描述\ascii{OP}存储或计算设备的具体位置。

\subsection{形式化}

一个设备规范可以形式化地描述为:

\begin{leftbar}
\begin{python}
DEVICE_SPEC ::= COLOCATED_NODE | PARTIAL_SPEC
COLOCATED_NODE ::= "@" NODE_NAME
PARTIAL_SPEC ::= ("/" CONSTRAINT) *
CONSTRAINT ::= ("job:" JOB_NAME)
             | ("replica:" [1-9][0-9]*)
             | ("task:" [1-9][0-9]*)
             | ( ("gpu" | "cpu") ":" ([1-9][0-9]* | "*") )
\end{python}
\end{leftbar}

\subsubsection{完整指定}

如下例所示,完整地描述某个\ascii{OP}被放置在\ascii{PS}作业,\ascii{0}号备份,\ascii{0}号任务,\ascii{GPU 0}号设备。

\begin{leftbar}
\begin{python}
/job:ps/replica:0/task:0/device:GPU:0
\end{python}
\end{leftbar}

\subsubsection{部分指定}

设备规范也可以部分指定,甚至为空。例如,下例仅描述了\code{GPU0}号设备。

\begin{leftbar}
\begin{python}
/device:GPU:0
\end{python}
\end{leftbar}

特殊地,当设备规范为空时,则表示对\ascii{OP}未实施设备约束,运行时自动选择设备放置\ascii{OP}。

\subsubsection{同位}

使用\code{COLOCATED\_NODE}指示该\ascii{OP}与指定的节点被同时放置在相同的设备上。例如,该节点与\code{other/node}放置在相同的设备上。

\begin{leftbar}
\begin{python}
@other/node  # colocate with "other/node"
\end{python}
\end{leftbar}

\subsubsection{DeviceSpec}

一个设备规范可以使用字符串,或者\code{DeviceSpec}表示。其中,\code{DeviceSpec}是一个值对象,使用如下\ascii{5}个标识确定设备规范。

\begin{enum}
  \eitem{作业名称}
  \eitem{备份索引}
  \eitem{任务索引}
  \eitem{设备类型}
  \eitem{设备索引}
\end{enum}

例如,使用\code{DeviceSpec}构造的设备规范。

\begin{leftbar}
\begin{python}
# '/job:ps/replica:0/task:0/device:CPU:0'
DeviceSpec(job="ps", replica=0, task=0, device_type="CPU", device_index=0)
\end{python}
\end{leftbar}

\subsection{上下文管理器}

常常使用上下文管理器\code{device(device\_spec)}指定\ascii{OP}设备规范,在该上下文的作用域内构造的\code{OP}在运行时将被放置在指定设备上运行。

\begin{leftbar}
\begin{python}
with g.device('/gpu:0'):
  # All OPs constructed here will be placed on GPU 0.
\end{python}
\end{leftbar}

其中,\code{device}是\code{Graph}的一个方法,它设计了一个栈式结构的上下文管理器,实现设备规范的闭包、合并、覆盖等特性。

\subsubsection{合并}

可以对两个不同范围的设备规范进行合并。

\begin{leftbar}
\begin{python}
with device("/job:ps"):
  # All OPs constructed here will be placed on PS.
  with device("/task:0/device:GPU:0"):
    # All OPs constructed here will be placed on
    # /job:ps/task:0/device:GPU:0
\end{python}
\end{leftbar}

\subsubsection{覆盖}

在合并两个相同范围的设备规范时,内部指定的设备规范具有高优先级,实现设备规范的覆盖。

\begin{leftbar}
\begin{python}
with device("/device:CPU:0"):
  # All OPs constructed here will be placed on CPU 0.
  with device("/job:ps/device:GPU:0"):
    # All OPs constructed here will be placed on
    # /job:ps/device:GPU:0
\end{python}
\end{leftbar}

\subsubsection{重置}

特殊地,当内部的设备规范置位为\code{None}时,将忽略外部所有设备规范的定义。

\begin{leftbar}
\begin{python}
with device("/device:GPU:0"):
  # All OPs constructed here will be placed on CPU 0.
  with device(None):
    # /device:GPU:0 will be ignored.
\end{python}
\end{leftbar}

\subsubsection{设备规范函数}

当指定设备规范时,常常使用字符串,或\code{DeviceSpec}进行描述。也可以使用更加灵活的\emph{设备规范函数},它提供了一种更加灵活的扩展方式指定设备规范。设备规范函数是一个回调函数,入参为\code{Operation},生成一个字符串格式的设备规范。

\begin{leftbar}
\begin{python}
def matmul_on_gpu(n):
 if n.type == "MatMul":
   return "/gpu:0"
 else:
   return "/cpu:0"

with g.device(matmul_on_gpu):
  # All OPs of type "MatMul" constructed in this context
  # will be placed on GPU 0; all other OPs will be placed
  # on CPU 0.
\end{python}
\end{leftbar}

\subsubsection{实现}

\code{Graph.device(spec)}实现了一个栈式结构的上下文管理器,它可以接受字符串格式的设备规范,或者\emph{设备规范函数}。事实上,当给\code{device}函数传递字符串,或者\code{DeviceSpec}时,首先会对该字符串,或\code{DeviceSpec}做一个简单的适配,统一转换为设备规范函数。

\begin{leftbar}
\begin{python}
class Graph(object):
  def device(self, device_name_or_func):
    def to_device_func():
      if (device_name_or_func is not None
          and not callable(device_name_or_func)):
        return pydev.merge_device(device_name_or_func)
      else:
        return device_name_or_func

    try:
      self._device_function_stack.append(to_device_func())
      yield
    finally:
      self._device_function_stack.pop()
\end{python}
\end{leftbar}

当用户使用\code{device}时,未显式地指定图实例,则隐式地使用全局唯一的默认图实例。也就是说,\code{tf.device(spec)}函数事实上是对\code{get\_default\_graph().device(spec)}的一个简单包装。

\begin{leftbar}
\begin{python}
# tensorflow/python/framework/ops.py
def device(device_name_or_function):
  return get_default_graph().device(device_name_or_function)
\end{python}
\end{leftbar}

\subsubsection{应用}

在\code{Graph.device}实现中,\code{pydev.merge\_device}生成一个设备规范函数。该设备函数以输入的\code{spec}创建新的副本,通过调用\code{copy\_spec.merge\_from(current\_device)},合并既有的\code{node\_def.device}设备规范,并且\code{node\_def.device}具有更高的优先级。实现较为隐晦,为什么\code{node\_def.device}具有较高优先级呢?这取决于\code{\_apply\_device\_functions}实现覆盖、合并、重置的需求。

\begin{leftbar}
\begin{python}
def merge_device(spec):
  # replace string to DeviceSpec
  if not isinstance(spec, DeviceSpec):
    spec = DeviceSpec.from_string(spec or "")

  # returns a device function that merges devices specifications
  def _device_function(node_def):
    current_device = DeviceSpec.from_string(node_def.device or "")
    copy_spec = copy.copy(spec)

    # IMPORTANT: `node\_def.device` takes precedence.
    copy_spec.merge_from(current_device)      
    return copy_spec
  return _device_function
\end{python}
\end{leftbar}

当\code{Graph.create\_op}时,将调用\code{\_apply\_device\_functions}设置\code{NodeDef}的设备规范。它将对\code{\_device\_function\_stack}依次实施出栈操作,并调用相应的设备规范函数,并将结果直接设置为\code{NodeDef}的设备规范。如此,内嵌指定的\code{device}具有较高优先级,实现了合并、覆盖、重置外围指定的\code{device}。

\begin{leftbar}
\begin{python}
class Graph(object):
  def _apply_device_functions(self, op):
    for device_function in reversed(self._device_function_stack):
      if device_function is None:
        break
      # IMPORTANT: `node\_def.device` takes precedence.  
      op._set_device(device_function(op)) 
\end{python}
\end{leftbar}

\end{content}
