\chapter{前言} 
\label{ch:preface}

\section*{本书定位}

\begin{content}

这是一本剖析\ascii{TensorFlow}内核工作原理的书籍,并非讲述如何使用\ascii{TensorFlow}构建机器学习模型,也不会讲述应用\ascii{TensorFlow}的最佳实践。本书将通过剖析\ascii{TensorFlow}源代码的方式,揭示\ascii{TensorFlow}的系统架构、领域模型、工作原理、及其实现模式等相关内容,以便揭示内在的知识。

\end{content}


\section*{面向的读者}

\begin{content}

本书假设读者已经了解机器学习相关基本概念与理论,了解机器学习相关的基本方法论;同时,假设读者熟悉\ascii{Python, C++}等程序设计语言。

本书适合于渴望深入了解\ascii{TensorFlow}内核设计,期望改善\ascii{TensorFlow}系统设计和性能优化,及其探究\ascii{TensorFlow}关键技术的设计和实现的系统架构师、\ascii{AI}算法工程师、和\ascii{AI}软件工程师。

\end{content}

\section*{阅读方式}

\begin{content}

初次阅读本书,推荐循序渐进的阅读方式;对于高级用户,可以选择感兴趣的章节阅读。首次使用\ascii{TensorFlow}时,推荐从源代码完整地构建一次\ascii{TensorFlow},以便了解系统的构建方式,及其理顺所依赖的基本组件库。

另外,推荐使用\ascii{TensorFlow}亲自实践一些具体应用,以便加深对\ascii{TensorFlow}系统行为的认识和理解,熟悉常见\ascii{API}的使用方法和工作原理。强烈推荐阅读本书的同时,阅读\ascii{TensorFlow}关键代码;关于阅读代码的最佳实践,请查阅本书附录\ascii{A}的内容。

\end{content}

\section*{版本说明}

\begin{content}

本书写作时,\ascii{TensorFlow}稳定发布版本为\ascii{1.2}。不排除本书讲解的部分\ascii{API}将来被废弃,也不保证某些系统实现在未来版本发生变化,甚至被删除。

同时,为了更直接的阐述问题的本质,书中部分代码做了局部的重构;删除了部分异常处理分支,或日志打印,甚至是某些可选参数列表。但是,这样的局部重构,不会影响读者理解系统的主要行为特征,更有利于读者掌握系统的工作原理。

同时,为了简化计算图的表达,本书中的计算图并非来自\ascii{TensorBoard},而是采用简化了的,等价的图结构。同样地,简化了的图结构,也不会降低读者对真实图结构的认识和理解。

\end{content}

% \section*{英语术语}

% \begin{content}

% 因为我所撰写的文章,是供相关专业人士阅读,而非科普读物,因此在书中保留专业领域中朗朗上口的英语术语,故意不做翻译。例如,书中直接使用\ascii{OP}的术语,而不是将其翻译为「操作」。

% 但是,这会造成大面积的中英混杂的表达方式。幸运的是,绝对部分所使用的英语术语都是名词,极少出现动词或者形容词。但是,无论如何都不会丢失原本的主体语义和逻辑。

% 万事都有例外,对于无歧义的,表达简短,且语义明确的术语,会使用中文术语表示。特殊地,对于中文术语表达存在歧义时,会同时标注中文术语和英语术语。例如,检查点(\ascii{Checkpoint}),协调器(\ascii{Coordinator})。

% 一般地,无歧义的中文术语表定义在\reftbl{glossary}。

% \begin{table}[!htb]
% \resizebox{0.95\textwidth}{!} {
% \begin{tabular*}{1.2\textwidth}{@{}ll@{}}
% \toprule
% \ascii{英文} & \ascii{中文} \\
% \midrule
% \ascii{Variable} & \ascii{变量,参数} \\
% \ascii{Session} & \ascii{会话} \\ 
% \ascii{Device} & \ascii{设备} \\ 
% \bottomrule
% \end{tabular*}
% }
% \caption{规范约定}
% \label{tbl:glossary}
% \end{table}

% \end{content}

% \section*{代码重构}

% \begin{content}

% 在\ascii{IDE}里阅读代码,可以借助于上下文信息协助程序员阅读代码。但在书中展示的源代码,由于篇幅受限,再加之缺乏上下文的缘故,代码阅读的效果将大大折扣。

% 一般地,为了更好地协助读者加快代码理解的速度和质量,常见的办法包括代码批注,内容讲解,算法描述,数据结构描述。但在作者的价值观中,与其批注过多的代码注释,不如重构代码,让代码直接表达语义。因此在本书中,作者另辟蹊径,尝试局部的代码重构,增加代码的可理解性。

% 例如,在\code{distributed\_runtime/master.cc}中,存在如下代码片段,上半部分的语义就是按照\code{session\_handle}线性查找相应的\code{MasterSession}示例。

% \begin{leftbar}
% \begin{c++}
% void Master::ExtendSession(
%     const ExtendSessionRequest* req,
%     ExtendSessionResponse* resp, MyClosure done) {
%   // Find master session by session handle.
%   mu_.lock();
%   MasterSession* session = nullptr;
%   session = gtl::FindPtrOrNull(sessions_, req->session_handle());
%   if (session == nullptr) {
%     mu_.unlock();
%     done(errors::Aborted(
%          "Session ", req->session_handle(), " is not found."));
%     return;
%   }
%   session->Ref();
%   mu_.unlock();

%   SchedClosure([session, req, resp, done]() {
%     Status status = ValidateExternalGraphDefSyntax(req->graph_def());
%     if (status.ok()) {
%       status = session->Extend(req, resp);
%     }
%     session->Unref();
%     done(status);
%   });
% }

% void Master::PartialRunSetup(
%     const PartialRunSetupRequest* req,
%     PartialRunSetupResponse* resp, MyClosure done) {
%   // Find master session by session handle.
%   mu_.lock();
%   MasterSession* session = nullptr;
%   session = gtl::FindPtrOrNull(sessions_, req->session_handle());
%   if (session == nullptr) {
%     mu_.unlock();
%     done(errors::Aborted(
%          "Session ", req->session_handle(), " is not found."));
%     return;
%   }
%   session->Ref();
%   mu_.unlock();

%   SchedClosure([this, session, req, resp, done]() {
%     Status s = session->PartialRunSetup(req, resp);
%     session->Unref();
%     done(s);
%   });
% }
% \end{c++}
% \end{leftbar}

% 如果在此处尝试提取函数,直接地表达原有的语义。重构后的效果,不仅删除了冗余的注释,增强了代码的可理解性,而且间接消除了两者之间的重复代码。当然,为了保证本书中的代码示例与社区代码仓库的一致,作者会争取将重构后的代码合入\ascii{master}分支。

% \begin{leftbar}
% \begin{c++}
% void Master::ExtendSession(
%     const ExtendSessionRequest* req,
%     ExtendSessionResponse* resp, MyClosure done) {
%   auto session = FindMasterSession(req->session_handle());
%   if (session == nullptr) {
%     done(AbortedError(req));
%     return;
%   }

%   SchedClosure([session, req, resp, done]() {
%     Status status = ValidateExternalGraphDefSyntax(req->graph_def());
%     if (status.ok()) {
%       status = session->Extend(req, resp);
%     }
%     session->Unref();
%     done(status);
%   });
% }

% void Master::PartialRunSetup(
%     const PartialRunSetupRequest* req,
%     PartialRunSetupResponse* resp, MyClosure done) {
%   auto session = FindMasterSession(req->session_handle());
%   if (session == nullptr) {
%     done(AbortedError(req));
%     return;
%   }

%   SchedClosure([this, session, req, resp, done]() {
%     Status s = session->PartialRunSetup(req, resp);
%     session->Unref();
%     done(s);
%   });
% }
% \end{c++}
% \end{leftbar}

% \end{content}

\section*{在线帮助}

\begin{content}

为了更好地与读者交流,已在\ascii{Github}上建立了勘误表,及其相关补充说明。由于个人经验与能力有限,在有限的时间内难免犯错。如果读者在阅读过程中,如果发现相关错误,请帮忙提交\ascii{Pull Request},避免他人掉入相同的陷阱之中,让知识分享变得更加通畅,更加轻松,我将不甚感激。

同时,欢迎关注我的简书。我将持续更新相关的文章,与更多的朋友一起学习和进步。

\begin{enum}
  \eitem{\ascii{Github: \script{\url{https://github.com/horance-liu/tensorflow-internals-errors}}}}
  \eitem{\ascii{简书:\script{\url{http://www.jianshu.com/u/49d1f3b7049e}}}}
\end{enum}

\end{content}

\section*{致谢}

\begin{content}

感谢我的太太刘梅红,在工作之余完成对本书的审校,并提出了诸多修改的意见。

\end{content}
