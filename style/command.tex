\newenvironment{enum}
{
  \begin{spacing}{1.2}
  % \begin{enumerate}[1.]
  \begin{enumerate}
    \setlength{\itemsep}{0pt} 
    \setlength{\itemindent}{2em}
    %\setlength{\listparindent}{2em}
}{%
  \end{enumerate}
  \end{spacing}
}

\newenvironment{nitemize}
{
  \begin{itemize}
    \setlength{\itemsep}{0pt} 
    \setlength{\itemindent}{2em}
}{%
  \end{itemize}
}


\newcommand{\suggest}[1]{
\tikzstyle{mybox} = [draw=black, very thick,
rectangle, rounded corners, inner sep=9pt, inner ysep=20pt]
\tikzstyle{fancytitle} =[fill=white, text=black, ellipse]
\noindent
\begin{tikzpicture}
\node [mybox] (box){%
\begin{minipage}{\textwidth}
\fangsong
#1
\end{minipage}
};
\node[fancytitle, right=10pt] at (box.north west) {\emph{建议}};
% \node[fancytitle, rounded corners] at (box.east) {$\clubsuit$};
\end{tikzpicture}
}

\newcommand{\notice}[1]{
\tikzstyle{mybox} = [draw=black, very thick,
rectangle, rounded corners, inner sep=9pt, inner ysep=20pt]
\tikzstyle{fancytitle} =[fill=white, text=black]
\noindent
\begin{tikzpicture}
\node [mybox] (box){%
\begin{minipage}{\textwidth}
\fangsong
#1
\end{minipage}
};
\node[fancytitle, right=10pt] at (box.north west) {\emph{注意}};
%\node[fancytitle, rounded corners] at (box.east) {$\clubsuit$};
\end{tikzpicture}
}

\newcommand{\tip}[1]{
\tikzstyle{mybox} = [draw=black, very thick,
rectangle, rounded corners, inner sep=9pt, inner ysep=20pt]
\tikzstyle{fancytitle} =[fill=white, text=black]
\noindent
\begin{tikzpicture}
\node [mybox] (box){%
\begin{minipage}{\textwidth}
\fangsong
#1
\end{minipage}
};
\node[fancytitle, right=10pt] at (box.north west) {\emph{提示}};
%\node[fancytitle, rounded corners] at (box.east) {$\clubsuit$};
\end{tikzpicture}
}

\newcommand\refch[1]{\ascii{第\ref{ch:#1}章(\nameref{ch:#1})}}
\newcommand\refsec[1]{\ascii{\ref{sec:#1}节(\nameref{sec:#1})}}

\newcommand\eitem[1]{\item {\itshape {#1}}}
\newcommand\cpp{\ascii{C\nobreak+\nobreak+}}
\newcommand\clang{\ascii{C}}
\newcommand\tf{\ascii{TensorFlow}}

\newcommand\quo[1]{“#1”}

\newcommand\percent[1]{\ascii{#1\%}}

\newcommand{\trans}{\emph{事务}}
\newcommand{\act}{\emph{操作}}
\newcommand{\seqact}{\emph{顺序操作}}
\newcommand{\conact}{\emph{并发操作}}
\newcommand{\atomact}{\emph{基本操作}}
\newcommand{\syncact}{\emph{同步操作}}
\newcommand{\asynact}{\emph{异步操作}}
\newcommand{\action}[1]{\emph{\ascii{\itshape\_\_#1}}}
\newcommand{\sigwait}{\action{sig\_wait}}
\newcommand{\sigsync}{\action{sig\_sync}}
\newcommand{\sigreply}{\action{sig\_reply}}
\newcommand{\timerprot}{\action{timer\_prot}}
\newcommand{\unknownevet}{\ascii{UNKNOWN\_EVENT}}
\newcommand{\transdsl}{\ascii{Transaction DSL}}
\newcommand{\oper}[1]{\ascii{Action#1}}
\newcommand{\protproc}{\ascii{prot\_procedure}}

%\newcommand{\code}[1]{\ascii{\small{\texttt{#1}}}}
\newcommand{\code}[1]{\ascii{\footnotesize{\texttt{#1}}}}
\newcommand{\script}[1]{\ascii{\scriptsize{\texttt{#1}}}}


\newcommand{\inlinetitle}[1]{\large{\emph{#1}}}

%\newcommand{\Email}{\begingroup \def\UrlLeft{<}\def\UrlRight{>} \urlstyle{tt}\Url}
%\def\mailto|#1|{\href{mailto:#1}{Email|#1|}}
\newcommand{\contrib}[2]{#1\quad{\small\quad\textit{#2}}\\[1ex]}


\newcommand{\upcite}[1]{\textsuperscript{\cite{#1}}}
